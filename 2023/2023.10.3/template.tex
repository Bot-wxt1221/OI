%!TEX program = xelatex
\documentclass[UTF8]{ctexbeamer}
\usetheme{Singapore}
\usecolortheme{seahorse}

\author{wxt1221}
\date {2023.10.3}
\title{线段树}
\begin{document}
	\frame{\titlepage}
	\begin{frame}
		\frametitle{Table of Contents - 目录}
		\tableofcontents
	\end{frame}
	
	\section{引入}

	\begin{frame}
		
		\frametitle{引入}
		你需要在时间复杂度为 $\mathcal{O}(log_n)$ 的时间复杂度下处理序列上以下操作:
		\begin{itemize}
			\item 对于所有的 $i \in [l,r]$ ,使 $a_i \to a_i+t$ ;

			\item 查询 $\sum \limits _{i=l} ^{r} a_i$。

		\end{itemize}

		对于序列上的操作,想要做到 $\mathcal O(log_n)$ 的时间复杂度。可以考虑分治,把整个区间分成两个部分,然后继续向下分,可以获得一个二叉树的结构。
	\end{frame}

	\section{分析}
	\begin{frame}
		\frametitle{查询}
		很明显,树高为 $\mathcal O(log_n)$ ,查询时间复杂度放在后面来说,并且可以发现,可以维护的操作不止区间和。
	\end{frame}

	\begin{frame}
		\frametitle{修改}
		修改的时间复杂度爆炸,但是如果使用一些 trick 就可以巧妙避免。在树上进行处理具有递归性质,访问节点 $i(i \not= 1)$ ,一定会访问 $\frac i 2$ ,也就是父亲节点,所以如果处理到某个节点需要对整个区间进行处理,就可以不用向下处理,而是查询到这个节点的时候再处理 ,时间复杂度同查询。
	\end{frame}

	\section{时间复杂度}
	\begin{frame}
		\frametitle{时间复杂度}
		从下到上以开区间来算,左右端点同时挪向父亲节点,整个区间大小最多会变小 $2$ ,一直向上挪到父亲节点所有的都统计到了,时间复杂度就是树高即 $\mathcal O(log_n)$ ,这也正是 zkw 线段树的实现方式。
	\end{frame}

	\section{推荐题目}
	\begin{frame}
		\frametitle{推荐题目}
			\begin{itemize} 
				\item \href{https://www.luogu.com.cn/problem/P3372}{\textcolor{blue}{P3372}}

				\item \href{https://www.luogu.com.cn/problem/P3373}{\textcolor{blue}{P3373}}

				\item \href{https://www.luogu.com.cn/problem/AT_abc322_f}{\textcolor{blue}{abc322f}}

				\item \href{https://www.luogu.com.cn/problem/P8818}{\textcolor{blue}{P8818}}
			\end{itemize}
	\end{frame}
\end{document}
